%!TEX TS-program = xelatex
%!TEX encoding = UTF-8 Unicode

% THIS TEMPLATE IS DATED 20111117

\documentclass[12pt]{scrartcl}
\usepackage[tmargin=1in, bmargin=1in, lmargin=0.5in, rmargin=0.5in]{geometry}
\geometry{letterpaper}
%\geometry{landscape} % Activate for for rotated page geometry
%\usepackage[parfill]{parskip} % Activate to begin paragraphs with an empty line rather than an indent
\usepackage{graphicx}
\usepackage{amssymb}
\usepackage[fleqn,reqno]{amsmath}
\renewcommand\minalignsep{2cm}
\usepackage{mathspec}			% replaces amssymb; do not use unicode-math
\usepackage{url}		 			% allows normal display of URLs
\usepackage{datetime}		% allows \currenttime
\usepackage{verbatim} % allows control of font-size of verbatim text (disadvantage: "\" must be replaced with "\(\backslash\)"
\usepackage{txfonts}				% allows use of \mathbb forms
	\normalfont
	\usepackage[T1]{fontenc}
	\usepackage[LY1]{fontenc}
	\usepackage{textcomp}


\usepackage{mdwtab}				% allows the use of footnotes in tables
\usepackage[hang,flushmargin]{footmisc}	% makes footnotes flush with margin, etc.
\usepackage{setspace}			% control of line spacing
\usepackage[normalem]{ulem}		% underlining (with ``normal emphasis'' retained)
\usepackage{alltt}
\usepackage{comment}				% allows
%\usepackage{paralist}			% enables more compact lists; use compactitem, compactenum, and compactdesc
%\usepackage{enumitem}			% enables more flexible spacing in lists
%\setitemize{itemsep=0pt}
%\setlist{nolistsep,leftmargin=0.1in}
%\setlist{nolistsep}
\usepackage{longtable}		% allows breaking of tables across pages with \begin{longtable}
\usepackage{tikz}			% allows much graphical control
\usepackage{pgflibraryshapes}	% allows use of ellipse in tikz

%\usepackage[italic,noendash]{mathastext}	% propagates document text font to math mode; use option italic to get italic text in math mode

\setromanfont[Scale=1,Mapping=tex-text]{TeX Gyre Termes}
\setsansfont[Scale=MatchLowercase,Mapping=tex-text]{TeX Gyre Heros Cn}

\DeclareGraphicsRule{.tif}{png}{.png}{`convert #1 `dirname #1`/`basename #1 .tif`.png}

%\newcommand\gr[3]{\raisebox{#3\height}{\scalebox{#2}{\includegraphics{#1}}}}
\newcommand\gr[2]{\scalebox{#2}{\includegraphics{#1}}}
\newcommand\sect[1]{\section*{#1}\markboth{#1}{#1}}
%\newcommand\subsect[1]{\subsection*{#1}\markboth{\thesection #1}{\thesection #1}}



\usepackage{fancyhdr}
\pagestyle{fancy}
\fancyhead{}
\fancyfoot{}
\fancyhead[RO]{Branner: Matrix Transformation Notes}
\fancyhead[LE]{Branner: Matrix Transformation Notes}
\fancyfoot[C]{\thepage}
%\headheight-14.5pt		% this value chosen in response to warning messages

\usepackage{scrextend}
%\usepackage[justification=centering,width=.7\textwidth]{caption}		% allows removal of figure numbers with \caption, and better formatting of captions
% package caption causes error with screxend:
%     Command \l@addto@macro already defined.
% see \setcapwidth[justification]{width} in screxend docs
\KOMAoption{captions}{belowfigure}
\setcapwidth[c]{.7\textwidth}			% positions whole caption
\setcapindent{0pt}
\addtokomafont{caption}{\centering}	% justification of caption text only
\renewcommand*{\figureformat}{}		% this removes figure number
\renewcommand*{\captionformat}{}		% this removes ``Figure''

\usepackage{textcomp}			% required for listings with upquote
\usepackage{listings}			% allows good verbatim text for code (only)
\lstset{
  language=Python,
  showstringspaces=false,
  formfeed=\newpage,
  upquote=true,
  breaklines=true,
  tabsize=4,
  commentstyle=\itshape,
  basicstyle=\ttfamily,
  morekeywords={models, lambda, forms}
}


\frenchspacing
\raggedright

\parskip12pt

\title{Matrix Transformation Notes}
\author{David Prager Branner}
\date{14 December, 2013}


\begin{document}
\maketitle

\section{Rotation}

Make a model of how the various kinds of points move when a matrix is rotated:

% side by side figures
% code after http://tex.stackexchange.com/a/1645/3935
\setcapwidth[c]{.2\textwidth}			% positions whole caption
\begin{figure}[htbp]
 \centering%
 \begin{minipage}[b]{.3\linewidth}
  \centering%
   \begin{tabular}{|>{\centering}p{.35in}|>{\centering}p{.35in}|>{\centering}p{.35in}|}\hline
    a & f & b \\ 
    (0,0) & (1,0) & (2,0) \\ \hline
    & e & \\ 
    \dots & (1,1) & \dots \\ \hline
    d & & c \\ 
    (0,2) & \dots & (2,2) \\ \hline
   \end{tabular}
  \caption{Initial state}
 \end{minipage}%
% \hfill%
 \begin{minipage}[b]{.05\linewidth}
  \centering%
   \(\longrightarrow\)
   \quad \\
   \quad \\
   \quad \\
   \quad \\
   \quad \\
 \end{minipage}%
% \hfill%
 \begin{minipage}[b]{.3\linewidth}
  \centering%
   \begin{tabular}{|>{\centering}p{.35in}|>{\centering}p{.35in}|>{\centering}p{.35in}|}\hline
    d &  & a \\ 
    (0,0) & \dots & (2,0) \\ \hline
    & e & f \\ 
    \dots & (1,1) & (2,1) \\ \hline
    c & \dots & b \\ 
    (0,2) & & (2,2) \\ \hline
   \end{tabular}
  \caption{After 90° rotation clockwise}
 \end{minipage}
\end{figure}
\setcapwidth[c]{.7\textwidth}			% positions whole caption

It may be useful to tabulate these single-point movements, to make your function easier to check:

\begin{table}[htdp]
 \begin{center}
  \begin{tabular}{|c|c|c|c|}\hline
   type & label & location & location \\ \hline
   corner & a & (0,0) & (2,0) \\ \hline
   corner & b & (2,0) & (2,2) \\ \hline
   corner & c & (2,2) & (0,2) \\ \hline
   corner & d & (0,2) & (0,0) \\ \hline
   center & e & (1,1) & (1,1) \\ \hline
   non-corner, non-center & f & (1,0) & (2,1) \\ \hline
  \end{tabular}
 \end{center}
\end{table}%

This rotation can be effected by applying the following code to every point in the matrix:

\begin{lstlisting}
def rotate_point(x, y, side_length):
    return ((side_length - 1 - y), x)
\end{lstlisting}

In the example above, \texttt{side\_length = 3}. Check the results of the function against the various edge cases.

%\newpage
%\subsection*{Counterclockwise rotation}
%\setcapwidth[c]{.2\textwidth}			% positions whole caption
%\begin{figure}[htbp]
% \begin{minipage}[b]{.3\linewidth}
% \centering%
%   \begin{tabular}{|>{\centering}p{.35in}|>{\centering}p{.35in}|>{\centering}p{.35in}|}\hline
%    a & f & b \\ 
%    (0,0) & (1,0) & (2,0) \\ \hline
%    & e & \\ 
%    \dots & (1,1) & \dots \\ \hline
%    d & & c \\ 
%    (0,2) & \dots & (2,2) \\ \hline
%   \end{tabular}
%  \caption{Initial state}
% \end{minipage}%
%% \hfill%
% \begin{minipage}[b]{.05\linewidth}
%  \centering%
%   \(\longrightarrow\)
%   \quad \\
%   \quad \\
%   \quad \\
%   \quad \\
%   \quad \\
% \end{minipage}%
%% \hfill%
% \begin{minipage}[b]{.3\linewidth}
%  \centering%
%   \begin{tabular}{|>{\centering}p{.35in}|>{\centering}p{.35in}|>{\centering}p{.35in}|}\hline
%    b & & c \\ 
%    (0,0) & \dots & (2,0) \\ \hline
%    f & e & \\ 
%    (0,1) & (1,1) & \dots \\ \hline
%    a & & d \\ 
%    (0,2) & \dots & (2,2) \\ \hline
%   \end{tabular}
%  \caption{After 90° rotation counterclockwise}
% \end{minipage}
%\end{figure}
%\setcapwidth[c]{.7\textwidth}			% positions whole caption
%
%This rotation can be effected by applying the following code to every point in the matrix:

Similarly, the following will bring about counterclockwise rotation:

\begin{lstlisting}
def rotate_point(x, y, side_length):
    return (y, (side_length - 1 - x))
\end{lstlisting}

%\begin{table}[htdp]
% \begin{center}
%  \begin{tabular}{|c|c|c|c|}\hline
%   type & label & location & location \\ \hline
%   corner & a & (0,0) & (0,2) \\ \hline
%   corner & b & (2,0) & (0,0) \\ \hline
%   corner & c & (2,2) & (2,0) \\ \hline
%   corner & d & (0,2) & (2,2) \\ \hline
%   center & e & (1,1) & (1,1) \\ \hline
%   non-corner, non-center & f & (1,0) & (0,1) \\ \hline
%  \end{tabular}
% \end{center}
%\end{table}%

\section{Flipping}

Flipping is a different kind of transformation from rotation, because it turns the ``face'' of the matrix away from us. We can flip along a row (``vertically''), a column, or a diagonal. Let's use the diagonal running from lower left to upper right. As before, make a model:

% side by side figures
% code after http://tex.stackexchange.com/a/1645/3935
\setcapwidth[c]{.2\textwidth}			% positions whole caption
\begin{figure}[htbp]
 \centering%
 \begin{minipage}[b]{.3\linewidth}
 \centering%
   \begin{tabular}{|>{\centering}p{.35in}|>{\centering}p{.35in}|>{\centering}p{.35in}|}\hline
    a & f & b \\ 
    (0,0) & (1,0) & (2,0) \\ \hline
    & e & \\ 
    \dots & (1,1) & \dots \\ \hline
    d & & c \\ 
    (0,2) & \dots & (2,2) \\ \hline
   \end{tabular}
  \caption{Initial state}
 \end{minipage}%
% \hfill%
 \begin{minipage}[b]{.05\linewidth}
  \centering%
   \(\longrightarrow\)
   \quad \\
   \quad \\
   \quad \\
   \quad \\
   \quad \\
 \end{minipage}%
% \hfill%
 \begin{minipage}[b]{.3\linewidth}
  \centering%
   \begin{tabular}{|>{\centering}p{.35in}|>{\centering}p{.35in}|>{\centering}p{.35in}|}\hline
    c &  & b \\ 
    (0,0) & \dots & (2,0) \\ \hline
     & e & f \\ 
    \dots & (1,1) & (2,1) \\ \hline
    d & & a \\ 
    (0,2) & \dots & (2,2) \\ \hline
   \end{tabular}
  \caption{After 90° rotation clockwise}
 \end{minipage}
\end{figure}
\setcapwidth[c]{.7\textwidth}			% positions whole caption

Tabulate for reference:

\begin{table}[htdp]
 \begin{center}
  \begin{tabular}{|c|c|c|c|}\hline
   type & label & location & location \\ \hline
   corner & a & (0,0) & (2,2) \\ \hline
   corner & b & (2,0) & (2,0) \\ \hline
   corner & c & (2,2) & (0,0) \\ \hline
   corner & d & (0,2) & (0,2) \\ \hline
   center & e & (1,1) & (1,1) \\ \hline
   non-corner, non-center & f & (1,0) & (2,1) \\ \hline
  \end{tabular}
 \end{center}
\end{table}%

Note that the points forming the diagonal on which we are flipping do not change position. All the other points move to a place symmetrically opposite their place of origin across the diagonal. 

\begin{lstlisting}
def flip_on_diag_lowerleft_to_upperright(x, y, side_length):
    # points change place only if not on this diagonal
    if x + y !== side_length:
        return side_length - 1 - y, side_length - 1 - x
    return x, y
\end{lstlisting}


\end{document}